\begin{DoxyParagraph}{}
The user of C\+U\+G\+AR needs to familiarize with the fact that on a G\+PU equipped system there is both a {\itshape host}, controlled by a {\itshape C\+PU}, and one or multiple {\itshape G\+PU} {\itshape devices}, with distinct memory spaces. Hence, there can be several types of functions and data-\/structures\+: 
\end{DoxyParagraph}
\begin{DoxyParagraph}{}

\begin{DoxyItemize}
\item single-\/threaded functions that can be called by a host thread
\item single-\/threaded functions that can be called by a device thread
\item single-\/threaded functions that can be called both on the host and the device
\item parallel functions that can be called by a host thread, and spawn one or more sets of host threads
\item parallel functions that can be called by a host thread, but spawn one or more sets of device threads 
\end{DoxyItemize}
\end{DoxyParagraph}
\begin{DoxyParagraph}{}

\begin{DoxyItemize}
\item data-\/structures that encapsulate host data and are meant to be used on the host (e.\+g. a resizable host vector, cugar\+::vector$<$host\+\_\+tag,\+T$>$)
\item data-\/structures that encapsulate device data but are meant to be used on the host (e.\+g. a resizable device vector, cugar\+::vector$<$device\+\_\+tag,\+T$>$)
\item data-\/structures that encapsulate device data and are meant to be used on the device 
\end{DoxyItemize}
\end{DoxyParagraph}
\begin{DoxyParagraph}{}
Unified Virtual Memory allows to use any data-\/structure anywhere, but for performance-\/oriented applications it can be beneficial to have explicit control of placement in the memory hierarchy.
\end{DoxyParagraph}
\hypertarget{host_device_page_PlainViewsSection}{}\section{Plain Views}\label{host_device_page_PlainViewsSection}
\begin{DoxyParagraph}{}
The fact that some data structures contain device data but can only be used from the host, coupled with the fact that at the moment C\+U\+DA does not allow to pass references as device kernel arguments and requires to pass P\+O\+Ds in, lends naturally to the definition of {\itshape plain views}\+: in C\+U\+G\+AR\textquotesingle{}s speech, a plain view of an object is essentially a {\itshape shallow reference} to an object\textquotesingle{}s data encapsulated in a P\+OD data structure that can be passed as kernel parameters. 
\end{DoxyParagraph}
\begin{DoxyParagraph}{}
C\+U\+G\+AR defines the generic function \hyperlink{group___basic_ga6eb01f34e803fa6b384bf9930f6db426}{plain\+\_\+view()} to obtain the plain view of a given object. Analogously it defines the meta function plain\+\_\+view\+\_\+subtype$<$\+T$>$\+::type to get the type of the plain view of any given type T (where defined). Moreover, as a convention C\+U\+G\+AR\textquotesingle{}s data structures T define the subtype T\+::plain\+\_\+view\+\_\+type and T\+::const\+\_\+plain\+\_\+view\+\_\+type to identify their plain view types. 
\end{DoxyParagraph}
\begin{DoxyParagraph}{}
As an example consider the following situation, where on the host you have created a large device vector you want to be filled by a device kernel. Ideally, you\textquotesingle{}d want to simply pass a reference to the vector to your kernel, as in\+: 
\begin{DoxyCode}
\_\_global\_\_ \textcolor{keywordtype}{void} my\_kernel(                   \textcolor{comment}{// the CUDA kernel}
    \hyperlink{structcugar_1_1vector}{cugar::vector<device\_tag,uint32>}& vec)   \textcolor{comment}{// ideally, receive a
       reference: doesn't work without UVM!}
\{
    \textcolor{keyword}{const} uint32 tid = threadIdx.x + blockIdx.x * blockDim.x; \textcolor{comment}{// compute a linear thread id}
    \textcolor{keywordflow}{if} (tid < vec.size())
        vec[tid] = tid * 10;
\}

\textcolor{keywordtype}{int} main()
\{
    \hyperlink{structcugar_1_1vector}{cugar::vector<device\_tag,uint32>} vec( 1000000 );

    \textcolor{keyword}{const} uint32 blockdim = 128;
    \textcolor{keyword}{const} uint32 n\_blocks = \hyperlink{group___basic_utils_gabb6714186dbbd864f0a9298944ba509b}{util::divide\_ri}( vec.size(), blockdim ); 
    my\_kernel<<<n\_blocks,blockdim>>>( vec );
\}
\end{DoxyCode}
 
\end{DoxyParagraph}
\begin{DoxyParagraph}{}
With U\+V\+M-\/capable G\+P\+Us this is technically possible, though it requires page migration. With C\+U\+G\+AR, you can do this instead\+: 
\begin{DoxyCode}
\_\_global\_\_ \textcolor{keywordtype}{void} my\_kernel(                   \textcolor{comment}{// the CUDA kernel}
    \hyperlink{structcugar_1_1vector__view}{cugar::vector\_view<uint32>} vec)          \textcolor{comment}{// CUGAR's surrogate of a reference}
\{
    \textcolor{keyword}{const} uint32 tid = threadIdx.x + blockIdx.x * blockDim.x; \textcolor{comment}{// compute a linear thread id}
    \textcolor{keywordflow}{if} (tid < vec.\hyperlink{structcugar_1_1vector__view_a773841d0e535b07e40a99891e22d937e}{size}())
        vec[tid] = tid * 10;
\}

\textcolor{keywordtype}{int} main()
\{
    \hyperlink{structcugar_1_1vector}{cugar::vector<device\_tag,uint32>} vec( 1000000 );

    \textcolor{keyword}{const} uint32 blockdim = 128;
    \textcolor{keyword}{const} uint32 n\_blocks = \hyperlink{group___basic_utils_gabb6714186dbbd864f0a9298944ba509b}{util::divide\_ri}( vec.\hyperlink{structcugar_1_1vector__view_a773841d0e535b07e40a99891e22d937e}{size}(), blockdim );
    my\_kernel<<<n\_blocks,blockdim>>>( \hyperlink{namespacecugar_a347f91de482f0cb8dcba21c086b0aa46}{cugar::plain\_view}( vec ) );
\}
\end{DoxyCode}
 
\end{DoxyParagraph}
\begin{DoxyParagraph}{}
This basic pattern can be applied to all of C\+U\+G\+AR\textquotesingle{}s data structures that are meant to be setup from the host and accessed from the device.
\end{DoxyParagraph}
Next\+: hello\+\_\+cugar\+\_\+page 