This module implements a priority queue adaptor, supporting O(log(\+N)) push/pop operations. Unlike std\+::priority\+\_\+queue, this class can be used both in host and device C\+U\+DA code\+:


\begin{DoxyItemize}
\item \hyperlink{structcugar_1_1priority__queue}{priority\+\_\+queue}
\end{DoxyItemize}\hypertarget{priority_queues_page_ExampleSection}{}\section{Example}\label{priority_queues_page_ExampleSection}

\begin{DoxyCode}
\textcolor{comment}{// build a simple priority\_queue over 4 integers}
\textcolor{keyword}{typedef} vector\_view<uint32*>                vector\_type;
\textcolor{keyword}{typedef} priority\_queue<uint32, vector\_type> queue\_type;

uint32 queue\_storage[4];

\textcolor{comment}{// construct the queue}
queue\_type queue( vector\_type( 0u, queue\_storage ) );

\textcolor{comment}{// push a few items}
queue.push( 3 );
queue.push( 8 );
queue.push( 1 );
queue.push( 5 );

\textcolor{comment}{// pop from the top}
printf( \textcolor{stringliteral}{"%u\(\backslash\)n"}, queue.top() );      \textcolor{comment}{// -> 8}
queue.pop();
printf( \textcolor{stringliteral}{"%u\(\backslash\)n"}, queue.top() );      \textcolor{comment}{// -> 5}
queue.pop();
printf( \textcolor{stringliteral}{"%u\(\backslash\)n"}, queue.top() );      \textcolor{comment}{// -> 3}
queue.pop();
printf( \textcolor{stringliteral}{"%u\(\backslash\)n"}, queue.top() );      \textcolor{comment}{// -> 1}
\end{DoxyCode}
 