\begin{DoxyParagraph}{}
This \hyperlink{group__bvh}{module} implements data-\/structures and functions to store, build and manipulate B\+V\+Hs.
\end{DoxyParagraph}

\begin{DoxyItemize}
\item \hyperlink{structcugar_1_1_bvh__node}{Bvh\+\_\+node}
\item \hyperlink{structcugar_1_1_bvh__node__3d}{Bvh\+\_\+node\+\_\+3d}
\item \hyperlink{structcugar_1_1_bvh}{Bvh}
\item \hyperlink{classcugar_1_1_bvh__builder}{Bvh\+\_\+builder}
\item \hyperlink{classcugar_1_1_bvh__sah__builder}{Bvh\+\_\+sah\+\_\+builder}
\item \hyperlink{structcugar_1_1cuda_1_1_l_b_v_h__builder}{cuda\+::\+L\+B\+V\+H\+\_\+builder}
\end{DoxyItemize}

\begin{DoxyParagraph}{}
As an example, consider the following code to create an L\+B\+VH tree over a set of points, in parallel, on the device\+: 
\begin{DoxyCode}
\textcolor{preprocessor}{#include <\hyperlink{lbvh__builder_8h}{cugar/bvh/cuda/lbvh\_builder.h}>}

thrust::device\_vector<Vector3f> points;
... \textcolor{comment}{// code to fill the input vector of points}

thrust::device\_vector<Bvh\_node> bvh\_nodes;
thrust::device\_vector<uint32>   bvh\_index;

\hyperlink{structcugar_1_1cuda_1_1_l_b_v_h__builder}{cugar::cuda::LBVH\_builder<uint64>} builder( &bvh\_nodes, &bvh\_index );
builder.build(
    Bbox3f( Vector3f(0.0f), Vector3f(1.0f) ),   \textcolor{comment}{// suppose all bboxes are in [0,1]^3}
    points.begin(),                             \textcolor{comment}{// begin iterator}
    points.end(),                               \textcolor{comment}{// end iterator}
    4 );                                        \textcolor{comment}{// target 4 objects per leaf}
\end{DoxyCode}
 
\end{DoxyParagraph}
